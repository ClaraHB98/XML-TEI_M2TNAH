\documentclass{article}
\usepackage{fancyhdr}
\usepackage{poemscol}
\usepackage{lmodern}


\begin{document}
\poemtitle{Mon rêve familier}
\begin{poem}
\setverselinemodulo{4}
\begin{stanza}
Je fais souvent ce rêve étrange et pénétrant\verseline
D'une femme inconnue, et que j'aime, et qui m'aime \verseline
Et qui n'est, chaque fois, ni tout à fait la même\verseline
Ni tout à fait une autre, et m'aime et me comprend.
\end{stanza}
\begin{stanza}
Car elle me comprend, et mon coeur, transparent \verseline
Pour elle seule, hélas ! cesse d'être un problème \verseline
Pour elle seule, et les moiteurs de mon front blême, \verseline
Elle seule les sait rafraîchir, en pleurant.
\end{stanza}
\begin{stanza}
Est-elle brune, blonde ou rousse ? - Je l'ignore. \verseline
Son nom ? Je me souviens qu'il est doux et sonore \verseline
Comme ceux des aimés que la Vie exila.\verseline
Son regard est pareil au regard des statues, \verseline
Et, pour sa voix, lointaine, et calme, et grave, elle a \verseline
L'inflexion des voix chères qui se sont tues.
\end{stanza}
\end{poem}
\attribution{Paul Verlaine}
\end{document}

